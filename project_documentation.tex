\documentclass{beamer}
\usetheme{Madrid}
\usecolortheme{default}
\usepackage{amsmath,amssymb,amsfonts}
\usepackage{graphicx}
\usepackage{hyperref}
\usepackage{algorithm}
\usepackage{algorithmic}
\usepackage{xcolor}
\usepackage{listings}

\hypersetup{
    colorlinks=true,
    linkcolor=blue,
    filecolor=magenta,      
    urlcolor=cyan
}

\title[PageRank Analysis]{Analyzing Email Network Influence:\\PageRank with Weighted Directed Graphs}
\author{Michelle Lawson \& Adriana Soldat}
\institute{Linear Algebra Final Project}
\date{\today}

\begin{document}

\begin{frame}
    \titlepage
\end{frame}

\begin{frame}
    \frametitle{Overview}
    \tableofcontents
\end{frame}

\section{Introduction}

\begin{frame}
    \frametitle{Project Overview}
    \begin{itemize}
        \item Analysis of the SNAP email-Eu-core dataset using PageRank
        \item Directed, weighted graph representing email communications
        \item Direction: who emails whom
        \item Weight: frequency of email communication
        \item Application of linear algebra for network analysis
        \item Community detection through Laplacian matrix
    \end{itemize}
\end{frame}

\begin{frame}
    \frametitle{Dataset: email-Eu-core}
    \begin{itemize}
        \item Email communication network from a European research institution
        \item Edge (u,v): person u sent person v at least one email
        \item Contains only internal communications
        \item Includes department affiliations (ground truth communities)
        \item 1005 nodes (institution members)
        \item 25571 edges (email communications)
        \item 42 departments (communities)
    \end{itemize}
    \small{Source: Stanford Network Analysis Project (SNAP)}
\end{frame}

\begin{frame}
    \frametitle{Project Objectives}
    \begin{itemize}
        \item Implement the PageRank algorithm for a weighted, directed email network
        \item Analyze how email frequency affects member influence
        \item Detect communities using the Laplacian matrix's eigenspaces
        \item Compare with ground-truth department affiliations
        \item Visualize the network with an intuitive heatmap representation
        \item Simulate changes to observe how new communications affect rankings
    \end{itemize}
\end{frame}

\section{Mathematical Framework}

\begin{frame}
    \frametitle{Weighted Adjacency Matrix}
    \begin{itemize}
        \item For a graph with $n$ nodes (institute members), we construct an $n \times n$ weighted adjacency matrix $W$
        \item Each entry represents email frequency:
    \end{itemize}
    \begin{equation}
        W_{ij} = \text{Number of emails from member $i$ to member $j$}
    \end{equation}
    \begin{itemize}
        \item If no emails exist between members, the corresponding matrix entry is 0
        \item The matrix is generally asymmetric, reflecting the directed nature of emails
    \end{itemize}
\end{frame}

\begin{frame}
    \frametitle{PageRank Algorithm}
    The PageRank vector $\mathbf{r}$ satisfies:
    \begin{equation}
        \mathbf{r} = \alpha M\mathbf{r} + (1-\alpha)\mathbf{p}
    \end{equation}
    
    Where:
    \begin{itemize}
        \item $\alpha$ is the damping factor (typically 0.85)
        \item $M$ is the transition matrix derived from the weighted adjacency matrix:
        $M_{ji} = \frac{W_{ij}}{\sum_k W_{ik}}$
        \item $\mathbf{p}$ is the personalization vector (uniform in our implementation)
    \end{itemize}
    
    Interpretation: The importance of a member depends on the importance of those who email them and the frequency of those emails.
\end{frame}

\begin{frame}
    \frametitle{Power Iteration Method}
    \begin{algorithm}[H]
    \caption{PageRank with Power Iteration}
    \begin{algorithmic}
    \STATE Initialize $\mathbf{r}_0$ to uniform distribution
    \STATE $k \gets 0$
    \WHILE{not converged}
        \STATE $\mathbf{r}_{k+1} \gets \alpha M \mathbf{r}_k + (1-\alpha)\mathbf{p}$
        \STATE $\mathbf{r}_{k+1} \gets \mathbf{r}_{k+1} / \|\mathbf{r}_{k+1}\|_1$
        \STATE $k \gets k + 1$
    \ENDWHILE
    \RETURN $\mathbf{r}_k$
    \end{algorithmic}
    \end{algorithm}
    
    Power iteration converges to the dominant eigenvector of the matrix.
\end{frame}

\begin{frame}
    \frametitle{Laplacian Matrix for Community Detection}
    The Laplacian matrix $L$ is defined as:
    \begin{equation}
        L = D - W_{sym}
    \end{equation}
    
    Where:
    \begin{itemize}
        \item $D$ is the degree matrix (diagonal matrix with node degrees)
        \item $W_{sym}$ is the symmetrized version of the weighted adjacency matrix
    \end{itemize}
    
    Key property: The multiplicity of the eigenvalue 0 in the spectrum of $L$ indicates the number of connected components in the graph.
\end{frame}

\section{Implementation}

\begin{frame}
    \frametitle{Implementation Overview}
    We developed both Python and MATLAB implementations:
    \begin{itemize}
        \item Python: Using NetworkX library for graph operations
        \item MATLAB: Using native matrix operations
        \item Core components in both implementations:
        \begin{itemize}
            \item Data loading from SNAP dataset files
            \item PageRank computation using power iteration
            \item Laplacian analysis for community detection
            \item Network visualization with heat map coloring
            \item Simulation of additional communications
        \end{itemize}
    \end{itemize}
\end{frame}

\begin{frame}[fragile]
    \frametitle{Python Implementation: PageRank}
    \begin{lstlisting}[language=Python, basicstyle=\tiny]
# PageRank implementation with custom weighting
def pagerank(G: nx.DiGraph, alpha: float = 0.85, tol: float = 1e-6,
             max_iter: int = 100) -> Dict[int, float]:
    return nx.pagerank(G, alpha=alpha, tol=tol, 
                     max_iter=max_iter, weight="weight")

# Display top-k nodes by PageRank
def display_topk(pr: Dict[int, float], k: int = 10) -> None:
    import pandas as pd
    top = sorted(pr.items(), key=lambda x: x[1], reverse=True)[:k]
    print(pd.DataFrame(top, columns=["node", "PageRank score"]).to_string(index=False))
    \end{lstlisting}
    
    [Insert screenshots of output showing top-ranked nodes]
\end{frame}

\begin{frame}[fragile]
    \frametitle{MATLAB Implementation: PageRank}
    \begin{lstlisting}[language=Matlab, basicstyle=\tiny]
% PageRank implementation
function pr = pagerank(G, weights, alpha, tol, max_iter)
    if nargin < 3, alpha = 0.85; end
    if nargin < 4, tol = 1e-6; end
    if nargin < 5, max_iter = 100; end
    
    n = size(G, 1);
    pr = ones(n, 1) / n;
    out_degree = sum(weights, 2);
    dangling = (out_degree == 0);
    
    for iter = 1:max_iter
        prev_pr = pr;
        M = weights ./ repmat(out_degree, 1, n);
        M(isnan(M)) = 0;
        
        rand_contribution = (alpha * sum(pr(dangling)) / n) + 
                           ((1 - alpha) / n);
        pr = alpha * M' * pr + rand_contribution * ones(n, 1);
        pr = pr / sum(pr);
        
        if norm(pr - prev_pr, 1) < tol
            break;
        end
    end
end
    \end{lstlisting}
\end{frame}

\begin{frame}[fragile]
    \frametitle{Community Detection Implementation}
    \begin{lstlisting}[language=Python, basicstyle=\tiny]
# Laplacian zero-eigenvalue multiplicity
def laplacian_zero_eigenspace_dim(G: nx.Graph) -> int:
    H = G.to_undirected(as_view=False)
    A = nx.to_scipy_sparse_array(H, weight="weight", dtype=float, format="csr")
    degrees = np.ravel(A.sum(axis=1))
    L = np.diag(degrees) - A.toarray()
    eigvals = np.linalg.eigvalsh(L)
    return int(np.sum(np.isclose(eigvals, 0.0, atol=1e-8)))
    \end{lstlisting}
    
    Our analysis found [insert number] connected components in the email network, while the ground truth has 42 departments.
\end{frame}

\section{Results}

\begin{frame}
    \frametitle{PageRank Scores Distribution}
    [Insert visualization of PageRank distribution across nodes]
    
    Key observations:
    \begin{itemize}
        \item The distribution follows a power law (as expected in social networks)
        \item Most members have relatively low influence
        \item A small number of highly influential members (opinion leaders)
        \item [Additional observations from your results]
    \end{itemize}
\end{frame}

\begin{frame}
    \frametitle{Most Influential Members}
    [Insert table or visualization of top-10 members by PageRank]
    
    Characteristics of influential nodes:
    \begin{itemize}
        \item Higher connectivity (both incoming and outgoing emails)
        \item Connect to other influential members
        \item Often bridge between different departments
        \item [Additional characteristics from your analysis]
    \end{itemize}
\end{frame}

\begin{frame}
    \frametitle{Community Detection Results}
    [Insert visualization comparing detected communities with ground truth]
    
    Findings:
    \begin{itemize}
        \item Laplacian analysis identified [X] distinct communities
        \item [Percentage] overlap with the 42 ground-truth departments
        \item Some departments form natural clusters in the network
        \item Cross-departmental communication creates bridges between communities
        \item [Additional findings from your analysis]
    \end{itemize}
\end{frame}

\begin{frame}
    \frametitle{Simulating Additional Communications}
    [Insert visualization showing ranking changes after simulation]
    
    Effects of adding 100 simulated emails between nodes [X] and [Y]:
    \begin{itemize}
        \item Node [X] increased in rank from [A] to [B]
        \item Influence propagated to neighbors of Node [Y]
        \item [Additional effects observed in your simulation]
    \end{itemize}
\end{frame}

\section{Visualization}

\begin{frame}
    \frametitle{Network Visualization - Initial State}
    [Insert network visualization with PageRank heatmap]
    
    Visualization details:
    \begin{itemize}
        \item Node color represents PageRank score (red = high, blue = low)
        \item Node size proportional to PageRank score
        \item Edge thickness represents communication frequency
        \item Layout algorithm: Force-directed placement
    \end{itemize}
\end{frame}

\begin{frame}
    \frametitle{Network Changes After Simulation}
    [Insert before/after visualization comparison]
    
    Visual comparison showing:
    \begin{itemize}
        \item Shift in importance after 100 additional emails
        \item Propagation of influence through the network
        \item Local vs. global effects of new communications
        \item [Additional visual insights]
    \end{itemize}
\end{frame}

\begin{frame}
    \frametitle{Department-based Community Visualization}
    [Insert visualization of communities by department]
    
    Visual insights:
    \begin{itemize}
        \item Color-coding by department shows organizational structure
        \item Inter-departmental vs. intra-departmental communication patterns
        \item Correlation between PageRank and position in organizational hierarchy
        \item [Additional insights from department visualization]
    \end{itemize}
\end{frame}

\section{Real-World Applications}

\begin{frame}
    \frametitle{Organizational Applications}
    Our analysis provides insights relevant to organizational dynamics:
    \begin{itemize}
        \item Identifying key information brokers within organizations
        \item Optimizing communication flow and reducing bottlenecks
        \item Improving team structures based on natural communication patterns
        \item Detecting informal leadership networks alongside formal hierarchies
        \item Supporting organizational restructuring decisions with data
        \item Knowledge management and succession planning for critical roles
    \end{itemize}
\end{frame}

\begin{frame}
    \frametitle{Broader Applications}
    The techniques extend beyond email networks to other contexts:
    \begin{itemize}
        \item Social media influence analysis
        \item Scientific collaboration networks
        \item Financial transaction networks
        \item Recommendation systems
        \item Epidemic spread modeling
        \item Supply chain optimization
    \end{itemize}
\end{frame}

\begin{frame}
    \frametitle{Limitations and Future Work}
    \begin{itemize}
        \item Limitations:
        \begin{itemize}
            \item Email frequency may not perfectly correlate with influence
            \item Lacks content analysis of communications
            \item Temporal dynamics not fully captured
        \end{itemize}
        \item Future directions:
        \begin{itemize}
            \item Incorporate temporal evolution of the network
            \item Compare with formal organizational hierarchy
            \item Analyze email content for sentiment and topics
            \item Explore alternative community detection methods
        \end{itemize}
    \end{itemize}
\end{frame}

\section{Conclusion}

\begin{frame}
    \frametitle{Conclusion}
    Our analysis of the email-Eu-core network demonstrates:
    \begin{itemize}
        \item PageRank effectively identifies influential members in an email network
        \item Email communication patterns reveal informal influence structures
        \item Laplacian analysis can detect communities that may differ from formal departments
        \item Small changes in communication patterns can significantly shift influence distribution
        \item Linear algebra provides powerful tools for analyzing organizational networks
    \end{itemize}
    
    Key takeaway: The mathematical framework of linear algebra offers valuable insights into the structure and dynamics of communication networks.
\end{frame}

\begin{frame}[allowframebreaks]
    \frametitle{References}
    \begin{thebibliography}{9}
    \bibitem{page1999} Page, L., Brin, S., Motwani, R., \& Winograd, T. (1999). The PageRank citation ranking: Bringing order to the web. Stanford InfoLab.
    
    \bibitem{fortunato2010} Fortunato, S. (2010). Community detection in graphs. Physics Reports, 486(3-5), 75-174.
    
    \bibitem{leskovec2014} Leskovec, J., \& Krevl, A. (2014). SNAP Datasets: Stanford Large Network Dataset Collection. \url{http://snap.stanford.edu/data}
    
    \bibitem{yin2017} Yin, H., Benson, A. R., Leskovec, J., \& Gleich, D. F. (2017). Local Higher-order Graph Clustering. In Proceedings of the 23rd ACM SIGKDD International Conference on Knowledge Discovery and Data Mining.
    
    \bibitem{vonluxburg2007} von Luxburg, U. (2007). A tutorial on spectral clustering. Statistics and Computing, 17(4), 395-416.
    \end{thebibliography}
\end{frame}

\begin{frame}
    \frametitle{Thank You!}
    \begin{center}
        \Huge{Questions?}
        
        \vspace{1cm}
        \large{Michelle Lawson \& Adriana Soldat}
    \end{center}
\end{frame}

\end{document}
